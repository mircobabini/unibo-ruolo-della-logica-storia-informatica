%% LyX 2.0.3 created this file.  For more info, see http://www.lyx.org/.
%% Do not edit unless you really know what you are doing.
\documentclass[english]{article}
\usepackage[T1]{fontenc}
\usepackage[latin9]{inputenc}
\setlength{\parskip}{\bigskipamount}
\setlength{\parindent}{0pt}
\usepackage{float}
\usepackage{babel}
\begin{document}

\title{Ruolo della logica nella storia dell'Informatica}


\author{Mirco Babini}


\date{7 Gennaio 2013}

\maketitle
\pagebreak{}

\tableofcontents{}

\pagebreak{}


\section{Introduzione}


\paragraph{Logica e formalismi nella natura umana}

Volendo effettuare una rapida panoramica di quello che � il ruolo
della logica nella storia dell'informatica non possiamo che considerare
come, il \textbf{pensiero logico}, proprio per la sua definizione,
sia alla base di tutto ci� che implica o � basato su un \textbf{percorso
di sviluppo e crescita} che si evolve adattandosi alle varie necessit�
di \textbf{risoluzione dei problemi} informativi dell'epoca in cui
ci contestualizziamo.

L'informatica, intesa come disciplina che studia le \textbf{tecniche,
le metodologie e i problemi} connessi con la \textbf{rappresentazione,
la memorizzazione, il reperimento e l'elaborazione automatica} dell'informazione,
implica la necessit� di strategia per assolvere a tutti gli ambiti
di studio in maniera automatica grazie ad un \textbf{procedimento
descritto in maniera esplicita e non ambigua}.

Questi procedimenti che ricalcano i processi di elaborazione, sono
descritti dagli \textbf{algoritmi}, a loro volta esplicitati dai cosidetti
\textbf{linguaggi di programmazione}, che non sono altro che il risultato
della ricerca di un \textbf{linguaggio formale che potesse descrivere
qualsiasi situazione in maniera rigorosa, espilicit� ed accessibile}.

Quindi la logica, che rende possibili e significativi i linguaggi
di programmazione, ha un ruolo fondamentale in questo contesto nello
sviluppo di applicazioni e programmi sempre pi� complessi ed � tutt'ora
alla base dei moderni software utilizzati da ognuno di noi.


\section{Storia}


\paragraph{Logica intrinseca}

Antropologicamente parlando, l'origine dell'abilit� linguistica, raggiunta
al termine di un processo evolutivo durato centinaia di migliaia di
anni, ha segnato una linea fondamentale per la logica evolutiva orientata
al miglioramento della vita e all'assolvere le necessit� pi� importanti
quali organizzazione e alimentazione. � il prodotto diretto del \textbf{senso
logico intrinseco nella natura umana che sfocia nella necessit� di
dialogo e condivisione} che ha generato nel corso dei secoli il linguaggio
naturale, il quale marca la strada per tre nuove fondamenta dell'uomo,
ossia il senso etico, il senso estetico e il \textbf{senso logico}.


\paragraph{Logica implicita}

A dimostrazione di questo nuovo senso logico, che pu� essere definito
come una sorta di \textbf{logica implicita} nell'uomo e che si distingue
dalla logica intrinseca poich� � esattamente da quest'ultima che ne
deriva, abbiamo una serie di riscontri pratici, ossia l'evoluzione
di metodi efficaci per utilizzare le risorse naturali e l'adozione
di strutture sociali complesse da cui ne deriva l'adozione di regole
adatte alla vita sociale. L'uomo ha quindi appreso una serie di competenze
nell'organizzare le proprie azioni ed � divenuto in grado di agire
razionalmente e logicamente sviluppando tecniche efficaci per interagire
con l'ambiente informativo circostante.

� il processo insito nella definizione di implicazione che permette
la comprensione di questo tipo di logica molto basilare e comune a
praticamente tutte le specie di animali del mondo. L'implicazione
consiste nella capacit� di rispondere alle sollecitazioni del mondo
esterno grazie a tre caratteristiche fondamentali che sono la sensibilit�,
la capacit� di far corrispondere una determinata risposta ad un determinato
stimolo e la capacit� di mettere in pratica le risposte appena citate.

Il concetto alla base di questa logica � l'intepretazione, ossia il
modo in cui vengono elaborate le risposte sulla base del significato
attribuito al segnale ricevuto.


\paragraph{Logica esplicita}

Il linguaggio naturale, inizialmente formato da suoni e gesti per
esprimere concetti nella forma parlata e di pittogrammi, che rappresentavano
oggetti, nella forma scritta, si � poi evoluta in fonogrammi per rappresentare
le parole e nelle definitiva lettere dell'alfabeto, componente fondamentale
di qualsiasi linguaggio, fatta eccezione per quelli iconografici.

La capacit� di combinare simboli e di trascriverli fornendo un significato
consono al contesto, come del resto anche la capacit� di contare,
sono tra le pi� antiche peculiarit� necessarie fin dall'antichit�
per poter gestire in maniera adeguata il proprio bestiame, i propri
cari e il commercio o baratto. Proprio per contare gli animali nasce
in definitiva il concetto di numero, evolutosi fino alla generazione
di un sistema complesso di calcolo, grazie a tutta una serie di connettivi
e simboli, denominato aritmetica.

I sistemi di scrittura e numerazione affiancati quindi alla matematica,
considerata la teoria dell'informazione, e combinati con la necessit�
di manipolare segni e idee fino al compimento di applicazioni pratiche,
permettono lo sviluppo di quella che � una \textbf{logica esplicita},
formale, rigorosa e in grado di convincere chiunque, che da vita a
quelli che comunemente conosciamo come linguaggi di programmazione.


\paragraph*{Linguaggi di programmazione}

Alla base dell'informatica vediamo l'ambito della filosofia, della
matematica e della logica come attivit� che, in sentesi con l'ambito
ingegneristico, ne danno una definizione sulla base di tre elementi
fondamentali che sono i manufatti (computer), i metodi (algoritmi)
e gli strumenti (linguaggi di programmazione).

Il percorso di crescita fino ad arrivare a questa definizione, inizia
nel diciassettesimo secolo quando \textbf{Leibniz} mise sostanzialmente
a punto la notazione utilizzata oggi giorno nel calcolo infinitesimale
sulla base dell'invenzione di Netwon avvenuta all'inizo della seconda
met� del 1600, la quale permette applicare il calcolo evitando ragionamenti
lunghi e complessi. Nel 1679 lo stesso Liebniz pubblica una serie
di scritti tra lui la \emph{characteristica universalis}, ossia un
sistema di segni con cui esprimere formalmente un pensiero e il \emph{calculus
ratiocinator}, ossia il metodo per manipolare e costruire ragionamenti
corretti.

� proprio dal \emph{calculus ratiocinator} di Leibniz che nasce l'idea
che deve esistere un modo per enumerare e produrre ragionamenti corretti
che evitino infinite discussioni grazie alla loro formalit� e che,
partendo da premesse corrette, fosse automatico enumerare tutte le
possilibi conclusioni corrette. Apre la strada al pensiero che si
possano fare calcoli generici con la logica tradizionale, pensiero
sviluppatosi verso la met� del diciannovesimo secolo grazie a \textbf{Boole},
il quale pubblica un sistema formale parziale (algebra di Boole) che
include alcune regole di deduzione, nel 1847.

Nel 1879 \textbf{Frege} riesce a dare vita ad un linguaggio artificiale
dotato di regole formali precise sia nel formare le frasi del linguaggio
che per utilizzarle al fine di dedurre altre frasi corrette, realizzando
parzialmente le idee li Leibniz pubblicate duecenti anni prima circa
la \emph{characteristica universalis}, e pubblicando Ideografia, il
volume contentente questi concetti. L'unica mancanza era il fatto
che, in caso di risultati non deducibili, non era lecito affermare
che non fosse possibile ottenerli in alcun caso, poich� non si conoscevano
metodi efficaci per dimostrare il contrario.

Frege � stato il primo fautore del logicismo, ossia della prospettiva
secondo la quale l'aritmetica, in quanto costituita da proposizioni
analitiche, sarebbe riducibile alla sola logica. Tratte queste conclusioni
pare che sia possibile sia definire un linguaggio formale capace di
parlare (quasi) di qualunque cosa (quello di Frege), che definire
in maniera rigorosa il ragionamento logico (con l'algebra di Boole),
aprendo le porte alla possibilit� di riscrivere nel nuovo linguaggio
quasi tutte le proposizioni corrette di qualsiasi ambito.


\paragraph{Paradosso, decisione e completezza}

Nel primo decennio del ventesimo secolo viene scoperto un \textbf{paradosso}
che mette in crisi i principi fondamentali dell'aritmetica e nasce
l'esigenza di evitare il verificarsi di questo paradosso dimostrando
che si pu� anche realizzare la formalizzazione completa di tutta la
matematica, che per� genera un problema fondamentale: garantire la
coerenza della formalizzazione evitando qualsiasi paradosso.

Il problema della \textbf{completezza}, ossia la dimostrazione che
ogni proposizione � vera (deducibile) o falsa (non deducibile) viene
risolto (parzialmente) da \textbf{Godel} che, ispirandosi al modello
del problema della terminazione diagonale di Cantor, pone dei limiti
alla completezza stessa e alla coerenza dei sistemi formali, ammettendo
quindi \textbf{proposizioni formalmente indecidibili}.

Il problema della decisione invece, ossia ricerca di un algoritmo
per decidere se ogni proposizione (costruibile correttamente) � vera
o falsa, viene affrontato da Turing nel 1936, il quale espone una
definizione formale di algoritmo come \emph{tutto e solo quello che
pu� fare in modo esplicito una macchina astratta} detta appunto macchina
di Turing. Per cercare di rendere pi� plausibile la definizione, Turing
genera la macchina universale con un insieme limitato di stati e regole
bene definiti e in grado di realizzare qualunque algoritmo. Questa
macchina pone le basi al concetto di distinzione tra costrutti e proposizioni
dell'algoritmo e dati, alla base dei \textbf{linguaggi di programmazione}.


\section{Conclusioni}


\paragraph{La nascita dei computer}

Da qui due grandi ingegneri del tempo, quali Shannon e Von Neumann,
dimostrano rispettivamente come l'algebra di Boole potesse, grazie
a \emph{relays} a due stati, essere alla base dei calcolatori automatici
in base binaria e come grazie ad una specifica architettura (appunto
di Von Neumann) potesse essere costruito un computer analogo alla
Macchina di Turing universale, quindi in grado di eseguire qualsiasi
algoritmo.
\end{document}
